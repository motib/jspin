\documentclass[11pt]{article}
\usepackage{mathpazo}
\usepackage{url}
\usepackage{graphicx}
\usepackage{verbatim}

\newcommand{\eri}{\textsc{Erigone}}
\newcommand{\prm}{\textsc{Promela}}
\newcommand{\eui}{\textsc{EUI}}
\newcommand{\p}[1]{\texttt{#1}}
\newcommand{\bu}[1]{\textsf{#1}}

\textwidth=15cm
\textheight=22cm
\topmargin=0pt
\headheight=0pt
\oddsidemargin=1cm
\headsep=0pt
\renewcommand{\baselinestretch}{1.1}
\setlength{\parskip}{0.20\baselineskip plus 1pt minus 1pt}
\parindent=0pt

\title{\eui{} - Graphical User Interface for \eri{}\\User's Guide\\
\mbox{}\\\large{Version 1.2.1}}
\author{Mordechai (Moti) Ben-Ari\\
Department of Science Teaching\\
Weizmann Institute of Science\\
Rehovot 76100 Israel\\
\textsf{http://stwww.weizmann.ac.il/g-cs/benari/}}
%\date{}
\begin{document}
\maketitle
\thispagestyle{empty}

\vfill

\begin{center}
Copyright \copyright{} 2009 by Mordechai (Moti) Ben-Ari.
\end{center}

This work is licensed under the Creative Commons Attribution-ShareAlike 3.0
License. To view a copy of this license, visit
\url{http://creativecommons.org/licenses/by-sa/3.0/}; or, (b) send a letter
to Creative Commons, 543 Howard Street, 5th Floor, San Francisco,
California, 94105, USA.

\newpage

\section{Introduction}

\eui{} is a graphical user interface for the \eri{} Model Checker that is
used for simulating and verifying concurrent programs. It is an adaptation
of the \textsc{jSpin} interface for \textsc{Spin}.
The user interface of \eui{} is simple, consisting of a single window
with menus, a toolbar and three adjustable text areas. \eri{} option
strings are automatically supplied and the \eri{} output is filtered and
formatted. All aspects of \eui{} are configurable: some at compile time,
some at initialization through a configuration file and some at runtime.

\subsection*{References}
\begin{itemize}
\item M. Ben-Ari. \textit{Principles of the Spin Model Checker}. Springer, 2008.
\item M. Ben-Ari. \textit{The \eri{} Model Checker}. \url{http://code.google.com/p/erigone/}.
\end{itemize}

\section{Installation and execution}
Install the Java SDK or JRE (\url{http://java.sun.com}).\footnote{The
default font for \eui{} is \p{Lucida Sans Typewriter}. This font may no longer be
available in the JRE you use. If you have the fonts from a previous version you
can copy them to the \p{lib/fonts} directory as explained in
\url{http://java.sun.com/j2se/1.5.0/docs/guide/intl/font.html}. Alternatively,
you can change the configuration file to use a monospaced font such as
\p{Courier} that is installed by default.} \eui{} needs Java 1.5 at least.

(For Windows) Download the \eui{} installation file called \p{eui-N.exe},
where \p{N} is the version number, and execute the installation file.

The installation will create the following subdirectories: \p{docs} for
the documentation, \p{eui} for the source files, \p{bin} for the
binaries of the \textsc{dot} tool, \p{txts} for the text files (help, about and
copyright), and \p{examples}.

\eui{} is also distributed as a \p{zip} file for use on non-Windows
systems. You will have to download \eri{} and \textsc{dot} yourself and
set the paths in the configuration file.

To run \eui{}, execute the command \p{javaw -jar EUI.jar}.
An optional argument names the \prm{} file to be opened initially.
A batch file \p{run.bat} is supplied which contains this command.

Configuration data (Appendix~\ref{a.cfg}) is in the file
\p{config.cfg}.
\textbf{When upgrading \eui{}, erase the configuration file before installing
a new version, so that new configuration options will be recognized.}
\eui{} searches for the configuration file in the current
directory; if it is not found, \eui{} searches for it in the directory
where the jar file is installed; if it is not found there, a new file
with default values is written.

To rebuild \eui{}, execute \p{build.bat}, which will compile all the source
files and create the file \p{EUI.jar} with the manifest file.

\section{\eui{} user interface}
The user interface includes a menu, a toolbar and three text areas.
Menu and toolbar commands have keyboard mnemonics or accelerators.
These can be easily configured by modifying the file \p{Config.java} and
rebuilding.

The left text area is used to display \prm{} source files. The bottom
text area is used to display messages from both \eri{} and \eui{}. The
right text area is used to display the output from \eri{}: statements
executed, values of variables and results of verifications. The text
areas can be resized by dragging the dividers and an area can be hidden
by clicking on the triangles in the dividers; the initial sizes can be
set in the configuration file. The toolbar button \bu{Maximize}
(\bu{Alt-M}) toggles between a normal split pane and a maximized right
text area that displays the \eri{} output.

\begin{figure}[tb]
\begin{center}
\includegraphics[width=140mm]{eui.png}
\end{center}
\end{figure}

\begin{description}
\item[\bu{File}] This menu includes selections for \bu{New}, \bu{Open}, 
\bu{Save}, \bu{Save As}, and \bu{Exit}. \bu{Switch file} closes the 
current file and opens the last file that was edited, if any.

\item[\bu{Edit}] This menu includes selections for \bu{Undo}, \bu{Redo}, 
\bu{Copy}, \bu{Cut}, \bu{Paste}, \bu{Find} and \bu{Find again}.

\item[\bu{Run}] This menu replicates the buttons on the toolbar for
running \eri{} in various modes. See the description in the next
section.

\item[\bu{Options}] The following \eri{} \bu{Limits} can be set (the
values are in thousands):
\begin{itemize}
\item \bu{Total steps}: The total number of steps in an execution of
\eri{}.
\item \bu{Progress steps}: A progress message will be displayed
after this number of steps of a verification.
\item \bu{State stack}, \bu{Location stack}: The size of the
verification stacks.
\end{itemize}

A nonzero value for \bu{Seed} will be used as the seed for generating
random numbers, enabling a random simulation to be repeated.

\bu{Default} restores the default values and \bu{Save} saves the current
options in the configuration file, together with the last directory from
which a source file was opened, and the current values of the splitpane
dividers. The file can be saved to the \bu{current} or \bu{install}
directory.

\item[\bu{Display}] This menu controls the display of the output in the
right text area.
\begin{itemize}
\item \bu{Maximize} replicates the button on the toolbar.
\item \bu{Save output} saves the contents of the text area in a file with
extension \p{.out}.

\item \bu{State space} runs \eri{} with the \p{-e} argument to generate
a diagram of the entire state space. The diagram is written in the
language of the \textsc{dot} tool of \textsc{GraphViz} and \textsc{dot}
is called to render the graph. If you first run a verification that
results in an error, the path of the counterexample will be emphasized
in the diagram.

\item The \bu{Trace} submenu is discussed in
Section~\ref{s.run}.
\end{itemize}

\item[\bu{Help}] \bu{Help} displays a short help file and \bu{About} 
displays copyright information.
\end{description}

\section{Running \eri{}}\label{s.run}
In the \bu{Run} menu and on the toolbar are eight selections for
running \eri{}. They all use the \prm{} source file that has been opened,
and save the file before execution.
During simulation and verification,
you can select \bu{Stop} to terminate the \eri{} process that has been forked.
\begin{description}
\item[\bu{Comple}] Runs the \prm{} compiler.
\item[\bu{Random}] Runs a random simulation.
\item[\bu{Interactive}] Runs an interactive simulation.
\item[\bu{Trail}] Runs a guided simulation using the trail
file created for a counterexample found in a verification.
\item[\bu{LTL2BA}] Translates an LTL formula to a B\"{u}chi automaton.
\item[\bu{Safety}] Runs a safety verification.
\item[\bu{Accept}] Runs an acceptance verification.
\item[\bu{Fairness}] Runs an acceptance verification with weak fairness.
\end{description}

\textbf{If you terminate \eui{} while \eri{} is running (for example by
entering \bu{ctrl-C} at the command line), make sure to terminate the
\eri{} process as well.} In Windows, press \bu{ctrl-alt-del}, \bu{Task
List} and \bu{Processes}. Select \bu{erigone.exe} and \bu{End Process}.

\subsection{Simulation}\label{s.sim}

\subsubsection{Interactive simulation}

During an interactive simulation, if there is more than one executable
statement or expression in a state, a dialog frame will pop up with the
current values of the variables and a list of the statements and
expressions that can be executed. The list can be displayed either as a
row of buttons, or---if there is not enough room---as a pulldown menu.
The choice of the format is determined by the value of the configuration
option \p{SELECT\_MENU}. There are also configuration options for
setting the width and height of the buttons or menu items.

The dialog can be navigated using the mouse; closing the dialog frame
will terminate interactive simulation. For keyboard navigation:
\begin{description}
\item[Buttons] \bu{Tab} and \bu{Shift-Tab} move through the buttons
and \bu{Space} selects the highlighted button. Press \bu{Esc} to terminate.
\item[Menu] Press the \bu{Down arrow} to display the list and to highlight the
item you want; press \bu{Return} to select it. Press \bu{Esc} to terminate.
\end{description}

\subsubsection{Filtered output}\label{s.filter}
The contents of the \eri{} output can be changed by selecting 
\bu{Display/Trace}. This pops up a dialog that can be used to set the
width of the field for the statement executed and the width of the field
for each variable. There are text areas for entering a list of strings defining 
variables to be \emph{excluded} from the display. Any variable containing 
a string from the list is not displayed; for example, \p{want} will 
exclude all variables that have \p{want} as a substring. If the variable name is 
prefixed by \p{+}, it will be included anyway. For example, if you have an 
array variable \p{test}, then the entries \p{test} and \p{+[1]} will 
exclude display of the elements of \p{test} except for \p{test[1]}. The 
list is saved in a file with extension \p{.exc}.

Similarly, a file of excluded statements can be created. The file 
extension is \p{.exs}. \bu{Exclude statements} should \emph{not} be used with 
interactive simulation.

\subsection{LTL formulas}
A correctness claim is specified by a \emph{Linear Temporal Logic (LTL)}
formula that is saved with extension \p{.prp}.
By default, when you open a \prm{} file,
\eui{} opens and loads a \p{.prp} file of the same name.
The file is saved whenever the source file is used.

Since \eri{} has no preprocessor, a defined atomic proposition cannot
be used; instead, an expression can be delimited by \verb+#+ as in
\verb+![]#(critical<=1)#+.

\eri{} requires a LTL formula for verification of \emph{acceptance}
(with or without fairness). A formula is optional for verifying safety;
if a formula is not needed, be sure to erase any data in the LTL field.

\newpage

\section{Software structure}

\p{EUI} is the main class and contains declarations
of the GUI components and instances
of the classes \p{Editor} and \p{RunSpin}.
Method \p{init} and the methods it calls initialize the GUI.
Method \p{action\-Per\-formed} is the event handler for all the menu items
and buttons; it calls the appropriate methods for handling each event.
The \bu{About} and \bu{Help} options are implemented by reading files
\p{copyright.txt} and \p{help.txt}, respectively, and displaying
them in the righthand pane.

\p{EUIFileFilter} is used with a \p{JFileChooser}
when opening and saving files: \prm{} source files,
LTL property files and \eri{} save output files.

\p{Limits} is the dialog frame for \bu{Options/Limits} and \p{Trace} is
the dialog frame for editing the list of variable strings to be excluded
from the display.

\p{Config} contains declarations of compile time configuration items.
Method \p{init} calls \p{set\-Default\-Properties} to initialize the instance
of \p{Properties} with the default values of the dynamic configuration
items; it then attempts to load the configuration file, and if unsuccessful,
the default properties are written to a new file.

\p{Editor} implements an editor using operations on a
\p{JTextArea}. It implements the interface \p{Document\-Listener} to flag
when the source has been modified. The class is also responsible
for the LTL formula \p{JTextField}. \p{EUI} calls method \p{writeFile}
to write \p{out} files, and method \p{readFile} to read
the text files to be displayed.

\p{LineNumbers} extends a \p{JComponent} to create line numbers
for the \p{RowHeaderView} of the editor \p{JScrollPane}
(thanks to Niko Myller for this code).

\p{UndoRedo} was extracted from an example on the Java web site.

The event handler in \p{EUI} calls \p{run} in class \p{RunSpin} to
execute \eri{}. \p{run} creates a thread of class \p{RunThread}, and
uses \p{ProcessBuilder} to set up the command, directory, merge the
output and error streams, and set up streams for I/O. The call
\p{runAndWait} is used for short calls like \bu{Compile}; this call does
not return until the completion of the subprocess. The call \p{run} will
return immediately after it has created the thread. In this case, the
event handler in \p{EUI} calls \p{isSpinRunning} to create a thread to
poll for termination of \eri{}; by creating a separate thread, the event
handler is freed to accept a selection of \bu{Stop}.

When more than one executable transition occurs during an interactive
simulation, the method \p{select} is called. This method pops up a
dialog to enable the user to make a selection. A \p{JFrame} is created
in a new thread of the inner class \p{SelectDialog} to wait for the
selection. \p{select} polls \p{selectedValue} which is set with the
selected button value or zero if the frame is closed or \bu{Esc}
pressed. In that case, \p{q} is sent to \eri{} to terminate the
simulation.

When generating the state space, after running \eri{} with the \p{-e}
argument, \textsc{dot} is called and its output directed to a file using
its \p{-o} argument. Class \p{DisplayImage} pops up a new Frame in
which the image is displayed.

\newpage

\appendix

\section{Configuration file}\label{a.cfg}

These tables give the properties in the configuration file and their
default values.

\begin{center}

\begin{tabular}{|p{.3\textwidth}|p{.4\textwidth}|}
\hline
\multicolumn{2}{|c|}{Directories and files}\\ \hline
\textsc{\ttfamily SOURCE\_DIRECTORY} & \verb+examples+ \\
\textsc{\ttfamily ERIGONE} &\verb+erigone.exe+ \\
\textsc{\ttfamily HELP\_FILE\_NAME} &\verb+txt\help.txt+\\
\textsc{\ttfamily ABOUT\_FILE\_NAME} &\verb+txt\copyright.txt+\\
\hline
\end{tabular}

\bigskip

\begin{tabular}{|p{.3\textwidth}|p{.4\textwidth}|}
\hline
\multicolumn{2}{|c|}{Options for executing \eri{}}\\ \hline
\textsc{\ttfamily COMPILE\_OPTIONS} &\verb+-c -dprv+\\
\textsc{\ttfamily RANDOM\_OPTIONS} &\verb+-r -dcmoprv+\\
\textsc{\ttfamily INTERACTIVE\_OPTIONS} &\verb+-i -dcemoprv+\\
\textsc{\ttfamily TRAIL\_OPTIONS} &\verb+-g -dcmoprv+\\
\textsc{\ttfamily LTL2BA\_OPTIONS} &\verb+-b -dbv+\\
\textsc{\ttfamily SAFETY\_OPTIONS} &\verb+-s -dgrv+\\
\textsc{\ttfamily ACCEPT\_OPTIONS} &\verb+-a -t -dgrv+\\
\textsc{\ttfamily FAIRNESS\_OPTIONS} &\verb+-f -t -dgrv+\\
\hline\hline
\textsc{\ttfamily TOTAL\_STEPS} & \verb+10+\\
\textsc{\ttfamily PROGRESS\_STEPS} & \verb+1+\\
\textsc{\ttfamily STATE\_STACK} & \verb+2+\\
\textsc{\ttfamily LOCATION\_STACK} & \verb+3+\\
\textsc{\ttfamily SEED} & \verb+0+\\
\textsc{\ttfamily SINGLE\_QUOTE} & \verb+false+\\\hline
\end{tabular}

\bigskip

\begin{tabular}{|p{.3\textwidth}|p{.4\textwidth}|}
\hline
\multicolumn{2}{|c|}{Trace options}\\ \hline
\textsc{\ttfamily PROCESS\_WIDTH} & \verb+7+\\
\textsc{\ttfamily STATEMENT\_WIDTH} & \verb+18+\\
\textsc{\ttfamily VARIABLE\_WIDTH} &\verb+10+\\
\textsc{\ttfamily LINES\_PER\_TITLE} &\verb+20+\\
\hline
\end{tabular}

\bigskip

\begin{tabular}{|p{.3\textwidth}|p{.4\textwidth}|}
\hline
\multicolumn{2}{|c|}{Text settings}\\ \hline
\textsc{\ttfamily WRAP} &\verb+true+\\
\textsc{\ttfamily TAB\_SIZE} &\verb+4+\\
\textsc{\ttfamily FONT\_FAMILY} & \verb+Lucida Sans Typewriter+\\ 
\textsc{\ttfamily FONT\_STYLE} & \verb+java.awt.Font.PLAIN+\\
\textsc{\ttfamily FONT\_SIZE} & \verb+14+\\\hline
\end{tabular}

\bigskip

\begin{tabular}{|p{.3\textwidth}|p{.4\textwidth}|}
\hline
\multicolumn{2}{|c|}{Frame size}\\ \hline
\textsc{\ttfamily WIDTH} &\verb+1000+\\
\textsc{\ttfamily HEIGHT} &\verb+700+\\\hline
\end{tabular}

\bigskip

\begin{tabular}{|p{.3\textwidth}|p{.4\textwidth}|}
\hline
\multicolumn{2}{|c|}{Interactive dialog settings}\\ \hline
\textsc{\ttfamily SELECT\_BUTTON} &\verb+120+\\
\textsc{\ttfamily SELECT\_HEIGHT} &\verb+70+\\
\textsc{\ttfamily SELECT\_MENU} &\verb+5+\\\hline
\end{tabular}

\bigskip

\begin{tabular}{|p{.3\textwidth}|p{.4\textwidth}|}
\hline
\multicolumn{2}{|c|}{Location of dividers}\\ \hline
\textsc{\ttfamily LR\_DIVIDER} &\verb+400+\\
\textsc{\ttfamily TB\_DIVIDER} &\verb+500+\\
\textsc{\ttfamily MIN\_DIVIDER} &\verb+50+\\\hline
\end{tabular}

\bigskip

\begin{tabular}{|p{.3\textwidth}|p{.4\textwidth}|}
\hline
\multicolumn{2}{|c|}{Delay while waiting for user input}\\ \hline
\textsc{\ttfamily POLLING\_DELAY} &\verb+200+\\\hline
\end{tabular}

\bigskip

\begin{tabular}{|p{.3\textwidth}|p{.4\textwidth}|}
\hline
\multicolumn{2}{|c|}{State space diagram options}\\ \hline
\textsc{\ttfamily SPACE\_OPTIONS} &\verb+-e+\\
\textsc{\ttfamily DOT} &\verb+bin\dot.exe+ \\
\textsc{\ttfamily DOT\_FORMAT} &\verb+png+\\\hline
\end{tabular}

\end{center}
\end{document}
